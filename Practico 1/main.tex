\documentclass[12pt,letterpaper]{article}
\usepackage[utf8]{inputenc}
\usepackage[spanish, es-tabla]{babel}
\usepackage[version=3]{mhchem}
\usepackage[journal=jacs]{chemstyle}
\usepackage{amsmath}
\usepackage{amsfonts}
\usepackage{amssymb}
\usepackage{makeidx}
\usepackage{esvect}
\usepackage{xcolor}
\usepackage[stable]{footmisc}
\usepackage[section]{placeins}
%Paquetes necesarios para tablas
\usepackage{longtable}
\usepackage{array}
\usepackage{xtab}
\usepackage{multirow}
\usepackage{colortab}
%Paquete para el manejo de las unidades
\usepackage{siunitx}
\sisetup{mode=text, output-decimal-marker = {,}, per-mode = symbol, qualifier-mode = phrase, qualifier-phrase = { de }, list-units = brackets, range-units = brackets, range-phrase = --}
\DeclareSIUnit[number-unit-product = \;] \atmosphere{atm}
\DeclareSIUnit[number-unit-product = \;] \pound{lb}
\DeclareSIUnit[number-unit-product = \;] \inch{"}
\DeclareSIUnit[number-unit-product = \;] \foot{ft}
\DeclareSIUnit[number-unit-product = \;] \yard{yd}
\DeclareSIUnit[number-unit-product = \;] \mile{mi}
\DeclareSIUnit[number-unit-product = \;] \pint{pt}
\DeclareSIUnit[number-unit-product = \;] \quart{qt}
\DeclareSIUnit[number-unit-product = \;] \flounce{fl-oz}
\DeclareSIUnit[number-unit-product = \;] \ounce{oz}
\DeclareSIUnit[number-unit-product = \;] \degreeFahrenheit{\SIUnitSymbolDegree F}
\DeclareSIUnit[number-unit-product = \;] \degreeRankine{\SIUnitSymbolDegree R}
\DeclareSIUnit[number-unit-product = \;] \usgallon{galón}
\DeclareSIUnit[number-unit-product = \;] \uma{uma}
\DeclareSIUnit[number-unit-product = \;] \ppm{ppm}
\DeclareSIUnit[number-unit-product = \;] \eqg{eq-g}
\DeclareSIUnit[number-unit-product = \;] \normal{\eqg\per\liter\of{solución}}
\DeclareSIUnit[number-unit-product = \;] \molal{\mole\per\kilo\gram\of{solvente}}
\usepackage{cancel}
%Paquetes necesarios para imágenes, pies de página, etc.
\usepackage{graphicx}
\usepackage{lmodern}
\usepackage{fancyhdr}
\usepackage[left=1cm,right=1cm,top=2cm,bottom=3cm]{geometry}

%Instrucción para evitar la indentación
%\setlength\parindent{0pt}
%Paquete para incluir la bibliografía
\usepackage[backend=bibtex,style=chem-acs,biblabel=dot]{biblatex}
\addbibresource{references.bib}

%Formato del título de las secciones

\usepackage{titlesec}
\usepackage{enumitem}
\titleformat*{\section}{\bfseries\large}
\titleformat*{\subsection}{\bfseries\normalsize}

%Creación del ambiente anexos
\usepackage{float}
\floatstyle{plaintop}
\newfloat{anexo}{thp}{anx}
\floatname{anexo}{Anexo}
\restylefloat{anexo}
\restylefloat{figure}

%Modificación del formato de los captions
\usepackage[margin=10pt,labelfont=bf]{caption}

%Paquete para incluir comentarios
\usepackage{todonotes}

%Paquete para incluir hipervínculos
\usepackage[colorlinks=true, 
            linkcolor = blue,
            urlcolor  = blue,
            citecolor = black,
            anchorcolor = blue]{hyperref}

%%%%%%%%%%%%%%%%%%%%%%
%Inicio del documento%
%%%%%%%%%%%%%%%%%%%%%%

\begin{document}
\renewcommand{\labelitemi}{$\checkmark$}

\renewcommand{\CancelColor}{\color{red}}

\newcolumntype{L}[1]{>{\raggedright\let\newline\\\arraybackslash}m{#1}}

\newcolumntype{C}[1]{>{\centering\let\newline\\\arraybackslash}m{#1}}

\newcolumntype{R}[1]{>{\raggedleft\let\newline\\\arraybackslash}m{#1}}

\begin{center}
	\textbf{\LARGE{Práctico 1 - Redes Neuronales 2020}}\\
	\vspace{7mm}
		\textbf{\large{FAMAF - UNC}}\\
	\vspace{4mm}
	\textbf{\large{Alumno: Pablo N. Rosa}}\\
	\textbf{\large{Profesor: Francisco A. Tamarit}}\\
\end{center}

\vspace{4mm}

\section*{\centering Resumen}

En este práctico se aplicará la teoría desarrollada para el análisis cualitativo de una ecuación diferencial no lineal. Para ello, se analizará el modelo presa-depredador de Lofka-Volterra, que consta de un sistema de ecuaciones diferenciales que no puede resolverse exactamente; por lo que serán necesarios métodos numéricos y cualitativos.

\section{Introducción}

Las ecuaciones de \textbf{Lotka-Volterra}, también conocidas como \textbf{ecuaciones predador-presa} o \textbf{presa-predador}, son un par de ecuaciones diferenciales de primer orden no lineales que se usan para describir dinámicas de sistemas biológicos en el que dos especies interactúan, una como \textit{presa} y otra como \textit{depredador}. Las ecuaciones fueron propuestas de forma independiente por \textit{Alfred J. Lotka} en 1925 y \textit{Vito Volterra} en 1926. 

Tales ecuaciones se definen como:

$$\dot{C}(t) = \alpha C(t) - \beta C(t)Z(t)$$
$$\dot{Z}(t) = -\gamma Z(t) + \delta C(t)Z(t)$$

La función $C(t)$ modela el
número de conejos (presa) en un ecosistema dado, y $Z(t)$ la cantidad de zorros (depredador) en el mismo ecosistema. 

La primera ecuación puede interpretarse la como el cambio del número de presas dado por su propio crecimiento menos la tasa de encuentros con predadores. Mientras que la segunda ecuación, se puede interpretar como el crecimiento de los depredadores por la caza de presas menos la muerte natural de estos.


\section{Resultados}

Para el caso que analizaremos los parámetros toman los siguientes valores:
$$ \alpha= 0.1 \quad \beta = 0.02, \quad \gamma = 0.3 \quad y \quad  \delta = 0.01$$

Además seguir la convención dada y simplicar la notación, se escribirá el sistema anterior prescidiendo del parámetro $t$, es decir, $C(t) = C$ y $Z(t) = Z$ por lo que el sistema resultante es:

$$\dot{C} =  0.1 C - 0.02 CZ $$
$$\dot{Z} = -0.3 Z + 0.01 CZ $$


Luego se linealiza el sistema, para lo cual se computa el Jacobiano.

$$A = 
\begin{bmatrix}
\dfrac{\partial \dot{C}}{\partial C} & \dfrac{\partial \dot{C}}{\partial Z} \\
 & \\
\dfrac{\partial \dot{Z}}{\partial C} & \dfrac{\partial \dot{Z}}{\partial Z} 
\end{bmatrix} 
= 
\begin{bmatrix}
0.1 - 0.02 Z & -0.02 C \\
0.01 Z & -0.3 + 0.01 C  
\end{bmatrix}
$$

Ahora el problema puede tratarse como uno lineal. El próximo paso consiste en buscar los puntos fijos,  es decir, los $(x^*, y^*)$ que simultáneamente satisfacen que:

$$\dot{C} = 0 \quad y \quad \dot{Z} = 0 $$

Notemos que:

$$0.1 C - 0.02 CZ = C(0.1 - 0.02 Z) = 0 \quad (1)$$
$$−0.3 Z + 0.01 CZ = Z(-0.3 - 0.01 C) = 0 \quad (2)$$

Por lo tanto un punto fijo es $P_1 = (0, 0)$. 

El otro punto fijo $P_2$, se obtiene por $(1)$ y $(2)$: 

$$C(0.1 - 0.02 Z) = 0 \iff (0.1 - 0.02 Z) = 0 \iff 0.1 = 0.02 Z \iff Z = \dfrac{0.1}{0.02} = 5$$ 
$$Z(-0.3 + 0.01 C) = 0 \iff (-0.3 + 0.01 C) = 0 \iff 0.3 = 0.01 C \iff C = \dfrac{0.3}{0.01} = 30 $$ 

Luego $P_2 = (30, 5)$. 


Ahora se clasifica cada punto:

La matriz evaluada en el punto $P_1 = (0, 0)$ es 
$$A = \begin{bmatrix}
0.1 & 0 \\
0 & -0.3   
\end{bmatrix}$$

Sus autovalores son: $\lambda_1 = 0.1$ y $\lambda_2 = -0.3$ y los autovectores correspondentes son: $\vv{v_1} = \begin{bmatrix}1 \\ 0\end{bmatrix}$ y $\vv{v_2} = \begin{bmatrix}0 \\ 1\end{bmatrix}$. Estos resultados muestran que este punto de equilibrio es un \textbf{saddle node}(\textit{punto de silla}) con soluciones que crecen exponencialmente a lo largo del eje $C$ y decrecen a lo largo del eje $Z$.



\subsection{Muestra de cálculos}



\section{Discusión de Resultados\label{sec:discusion}}

\subsection{Corrección de los puntos de ebullición}

Los puntos de ebullición fueron corregidos utilizando la ecuación de Sydney-Young\autocite{young:1902} (Ecuación \ref{eq:bpcorrection}). En donde \textit{t} es el punto de ebullición experimental, $C = \dfrac{dP}{dt} \times \dfrac{1}{T}$ es un valor aproximadamente constante que puede aproximarse a \num{0.00012} para líquidos polares y a \num{0.00010} para líquidos no polares y \textit{p} es la presión atmosférica del laboratorio expresada en mmHg.

\begin{equation}
\label{eq:bpcorrection}
\Delta T = C(760 - p)(273 + t)
\end{equation}

Se encontraron desviaciones respecto a los valores experimentales reportados en la literatura como se observa en la \ref{tab:boiling}. La diferencia observada se le puede atribuir al menos a tres razones; primero la presión atmosférica del laboratorio en el momento del experimento no se pudo medir así que se tomó un valor de \SI{680}{\mmHg}. Segundo el termómetro que se utilizó no fue calibrado previamente además que era un termómetro de inmersión completa y en el experimento solo estaba sumergido el bulbo, por lo que no existe certeza sobre la exactitud de la medida. Finalmente, el valor hay que recordar que los valores de la constante \textit{C} son aproximados y que en realidad esta constante es diferente para cada líquido.

\subsection{Ecuaciones Químicas}

A continuación se muestra un ejemplo de una reacción química escrita con la ayuda del paquete \textit{mhchem}.

\begin{center}
\ce{1/2 O2 _{(g)} + 2 H+ + 2 e^- <=> H2O} \hspace{19mm} $\times 3$ \hspace{20mm} $E^0 = \SI{1.229}{\volt}$\\
\vspace{3mm}
\ce{2 Cr^{3+} + 7 H2O <=> Cr2O7^{2-} + 14 H+ + 6 e^-} \hspace{24mm} $E^0 = \SI{-1.36}{\volt}$\\
\underline{\hspace{145mm}}

\ce{3 O2 _{(g)} + 4 Cr^{3+} + 8 H2O <=> 2 Cr2O7^{2-} + 16 H+ } \hspace{15mm} $E^0 = \SI{-0.13}{\volt}$\\
\end{center}

\section{Conclusiones\label{sec:conclusions}}

\SI{0.100}{\Molar} \\



\SI{23}{\gram}\\

$\SI{10d-2}{\cancel\gram\of{\ce{Na2SO4}}} \times \dfrac{\SI{1}{\mole}}{\SI{80}{\cancel\gram\of{\ce{Na2SO4}}}} = $

\num{3.1415592d-3}

\begin{equation}
\label{eq:ensayo}
x=3y+5
\end{equation}

Ahora quiero referenciar la ecuación \ref{eq:ensayo} y continuo escribiendo

\ce{H2SO4 + Ca(OH)2 -> CaSO4 v + 2 H2O ^}

\begin{equation}
\label{eq:cuadratica}
x=\frac { -b\pm \sqrt [ 2 ]{ { b }^{ 2 }-4ac }  }{ 2a } 
\end{equation}

Si necesito referirme a la ecuación cuadrática (ver ecuación \ref{eq:cuadratica}) y continúo escribiendo


\section{Procedimiento Experimental\label{sec:procedure}}

\section{Referencias\label{sec:references}}


\printbibliography[heading=none]
		
\end{document}
